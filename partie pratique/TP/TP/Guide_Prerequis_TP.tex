\documentclass[12pt,a4paper]{article}

% ============= PACKAGES =============
\usepackage[utf8]{inputenc}
\usepackage[french]{babel}
\usepackage[T1]{fontenc}

% Police académique et professionnelle : Times (standard pour documents académiques)
\usepackage{times}
\usepackage{geometry}
\usepackage{graphicx}
\usepackage{xcolor}
\usepackage{fancyhdr}
\usepackage{titlesec}
\usepackage{enumitem}
\usepackage{hyperref}
\usepackage{float}
\usepackage{tcolorbox}
\usepackage{listings}
\usepackage{tikz}
\usepackage{amssymb}

% ============= PAGE GEOMETRY =============
\geometry{
    left=2.5cm,
    right=2.5cm,
    top=2.5cm,
    bottom=2.5cm
}

% ============= COLORS (GUIDE VARIANT) =============
% Palette différente du TP principal tout en gardant la même structure visuelle
\definecolor{maincolor}{RGB}{88,24,69}      % Violet profond
\definecolor{secondcolor}{RGB}{0,150,136}   % Vert turquoise
\definecolor{graytext}{RGB}{80,80,80}
\definecolor{lightgray}{RGB}{240,240,240}
\definecolor{codegreen}{RGB}{0,128,0}
\definecolor{codegray}{RGB}{128,128,128}
\definecolor{codepurple}{RGB}{153,0,153}
\definecolor{backcolour}{RGB}{248,248,248}

% Couleurs spécifiques pour la page de garde du guide
\definecolor{bleuGuide}{RGB}{88,24,69}      % reprend maincolor
\definecolor{accentGuide}{RGB}{0,150,136}   % reprend secondcolor
\definecolor{jauneGuide}{RGB}{255,202,40}
\definecolor{grisclairGuide}{RGB}{215,215,215}
\definecolor{grisGuide}{RGB}{120,120,120}

% ============= HYPERLINKS =============
\hypersetup{
    colorlinks=true,
    linkcolor=maincolor,
    filecolor=maincolor,
    urlcolor=secondcolor,
    citecolor=maincolor,
    pdftitle={Guide des Prérequis - TP API Gateway et Microservices},
    pdfauthor={Mouad BENCAID},
    pdfsubject={Guide de préparation de l'environnement},
    pdfkeywords={Docker, Microservices, API Gateway, Middleware, TP}
}

% ============= HEADERS & FOOTERS =============
\pagestyle{fancy}
\fancyhf{}
\fancyhead[L]{\small\textcolor{graytext}{Guide des Prérequis}}
\fancyhead[R]{\small\textcolor{graytext}{TP - Middleware}}
\fancyfoot[C]{\thepage}
\renewcommand{\headrulewidth}{0.4pt}
\renewcommand{\footrulewidth}{0.4pt}

% ============= SECTION FORMATTING =============
\titleformat{\section}
    {\normalfont\Large\bfseries\color{maincolor}}
    {\thesection}{1em}{}[\titlerule]

\titleformat{\subsection}
    {\normalfont\large\bfseries\color{maincolor}}
    {\thesubsection}{1em}{}

\titleformat{\subsubsection}
    {\normalfont\normalsize\bfseries\color{secondcolor}}
    {\thesubsubsection}{1em}{}

% ============= CODE LISTINGS =============
\lstdefinestyle{bashstyle}{
    backgroundcolor=\color{backcolour},
    commentstyle=\color{codegreen},
    keywordstyle=\color{maincolor}\bfseries,
    numberstyle=\tiny\color{codegray},
    stringstyle=\color{codepurple},
    basicstyle=\ttfamily\small,
    breakatwhitespace=false,
    breaklines=true,
    captionpos=b,
    keepspaces=true,
    numbers=left,
    numbersep=5pt,
    showspaces=false,
    showstringspaces=false,
    showtabs=false,
    tabsize=2,
    frame=single,
    rulecolor=\color{lightgray}
}
\lstset{style=bashstyle}

% ============= CUSTOM BOXES =============
\newtcolorbox{infobox}[1][]{
    colback=accentGuide!5!white,
    colframe=accentGuide,
    fonttitle=\bfseries,
    title=#1,
    arc=2mm
}

\newtcolorbox{warningbox}[1][]{
    colback=jauneGuide!10!white,
    colframe=jauneGuide!80!black,
    fonttitle=\bfseries,
    title=#1,
    arc=2mm
}

\newtcolorbox{successbox}[1][]{
    colback=green!5!white,
    colframe=green!70!black,
    fonttitle=\bfseries,
    title=#1,
    arc=2mm
}

% ============= DOCUMENT START =============
\begin{document}

% ============= COVER PAGE =============
\begin{titlepage}
    % Fond graphique (même structure que le TP, nouvelle palette)
    \begin{tikzpicture}[remember picture,overlay]
        % Bandes latérales
        \fill[bleuGuide!95!black] (current page.north west) rectangle
            ([xshift=1.8cm]current page.south west);
        \fill[accentGuide!80!black] (current page.north east) rectangle
            ([xshift=-1.4cm]current page.south east);

        % Liseré accentué à gauche
        \fill[jauneGuide] ([xshift=1.5cm]current page.north west) rectangle
            ([xshift=1.65cm]current page.south west);

        % Bandes diagonales fines et parallèles (droite)
        \begin{scope}[shift={(current page.south east)},rotate=66]
            \fill[grisclairGuide] (0,0) rectangle (0.35cm,15cm);
        \end{scope}
        \begin{scope}[shift={(current page.south east)},rotate=66]
            \fill[accentGuide] (-0.8cm,0.2cm) rectangle (-0.45cm,15.2cm);
        \end{scope}
        \begin{scope}[shift={(current page.south east)},rotate=66]
            \fill[jauneGuide] (-1.6cm,0.4cm) rectangle (-1.25cm,15.4cm);
        \end{scope}
    \end{tikzpicture}

    {\sffamily
    % Légère augmentation de l'espace avant le titre
    \vspace*{2.0cm}

    % Titre principal
    \noindent
    {\Large\bfseries\color{bleuGuide!80!black}\MakeUppercase{Guide des Prérequis}}\\[1.2cm]
    {\fontsize{34}{40}\selectfont\bfseries TP Middleware \\[0.15cm] API Gateway \\[0.15cm] et Microservices}\\[1.0cm]
    {\large\bfseries\color{bleuGuide!80!black}Préparation de l'environnement Docker et Git avant la séance}\\[1.2cm]
    {\Large\itshape\color{grisGuide}Pour suivre la simulation en temps réel, sans temps morts}

    \vspace{3.2cm}

    % Section Auteur
    \begin{flushright}
        {\large\bfseries\color{bleuGuide!80!black}Préparé par :}\\[0.3cm]
        {\large\bfseries Mouad BENCAID}\\[0.1cm]
        {\large\bfseries Raissouni Abdellah}
    \end{flushright}

    \vfill

    % Pied de page
    \begin{flushright}
        {\normalsize\bfseries Année universitaire : 2025/2026}\\[0.2cm]
    \end{flushright}

    \vspace*{0.3cm}
    }% fin style sans empattement
\end{titlepage}

% On commence le contenu sans table des matières pour rester dans 4 pages
\thispagestyle{empty}

% ================= PAGE 2 : UTILITÉ DU GUIDE =================
\section*{Pourquoi ce guide ?}
\addcontentsline{toc}{section}{Pourquoi ce guide ?}

Ce document est \textbf{un guide de préparation} à utiliser \textbf{avant} la séance de Travaux Pratiques sur les middlewares et l'API Gateway.\par
\medskip

Son objectif est simple : faire en sorte que, le jour de la démonstration, \textbf{tout le monde puisse suivre la simulation en temps réel}, sans perdre de temps à télécharger des images Docker ou à installer des dépendances.

\begin{infobox}[\large Objectifs de ce guide]
Ce guide vous permet de :
\begin{itemize}[leftmargin=*]
    \item[$\checkmark$] Préparer un environnement technique stable pour pouvoir \textbf{observer concrètement le rôle des middlewares} dans une architecture microservices ;
    \item[$\checkmark$] Vérifier que votre environnement \textbf{supporte Docker Desktop et Git} pour simuler des \textbf{topologies de middleware} proches d'un contexte d'entreprise engagée ;
    \item[$\checkmark$] \textbf{Installer et tester Docker} (daemon en marche, commandes de base) afin de pouvoir démarrer et arrêter facilement les différentes topologies ;
    \item[$\checkmark$] \textbf{Cloner les deux repositories} utilisés pendant le TP : architecture \textbf{sans} middleware d'entrée et architecture \textbf{avec} middleware (API Gateway) ;
    
\end{itemize}
\end{infobox}

\begin{warningbox}[Ce que ce guide \textbf{n'est pas}]
Ce document \textbf{ne remplace pas} :
\begin{itemize}[leftmargin=*]
    \item le contenu théorique de l'atelier sur les middlewares ;
    \item le support TP principal (\og Guide de Travaux Pratiques \fg{}) que vous utiliserez pendant la séance ;
    \item les slides de présentation.
\end{itemize}
Il est uniquement destiné à \textbf{préparer votre machine} pour éviter les problèmes techniques de dernière minute.
\end{warningbox}

\vspace{0.5cm}

\begin{successbox}[Résultat attendu avant de venir en TP]
À l'issue de ce guide, sur votre poste :
\begin{itemize}[leftmargin=*]
    \item Docker Desktop doit être installé, lancé et testé ;
    \item Git doit être installé et opérationnel ;
    \item les deux projets doivent être clonés en local ;
    \item les images Docker des microservices doivent être \textbf{déjà construites} pour les deux architectures.
\end{itemize}
Ainsi, pendant la séance, vous n'aurez plus qu'à \textbf{lancer les conteneurs} à partir d'images déjà prêtes.
\end{successbox}

\newpage

% ================= PAGES 3-4 : SETUP =================
\section{Préparation de l'environnement}

\subsection{Prérequis matériels et système}

\begin{itemize}[leftmargin=*]
    \item[$\bullet$] Connexion Internet \textbf{temporaire} pour télécharger Docker, Git et les repositories (non nécessaire le jour du TP si tout est déjà fait) ;
    \item[$\bullet$] Système compatible Docker Desktop (Windows 10/11 64 bits ou macOS, ou bien une distribution GNU/Linux avec Docker Engine) ;
    \item[$\bullet$] Droits administrateur sur la machine (installation de logiciels).
\end{itemize}

\subsection{Étape 1 -- Installation et vérification de Docker}

\begin{enumerate}[leftmargin=*, label=\textbf{1.\arabic*)}]
    \item Rendez-vous sur le site officiel :
    \begin{itemize}[leftmargin=*]
        \item Windows / macOS : \url{https://www.docker.com/products/docker-desktop/}
    \end{itemize}

    \item Installez Docker Desktop en suivant l'assistant d'installation (sur Windows, activez WSL2 si nécessaire).

    \item Redémarrez votre machine si l'installateur le demande, puis lancez Docker Desktop et attendez que le daemon soit \og Up and running \fg{}.

    \item Ouvrez un terminal (PowerShell, CMD ou terminal Linux/macOS) et tapez :
\begin{lstlisting}[language=bash,caption={Vérification de l'installation Docker}]
docker --version
docker compose version   # ou: docker-compose --version
\end{lstlisting}

    \item Vérifiez qu'aucune erreur de type \og command not found \fg{} n'apparaît.
\end{enumerate}

\begin{warningbox}[En cas de problème Docker]
Si Docker ne démarre pas ou si les commandes ci-dessus échouent, résolvez le problème avant de passer à la suite (mise à jour de Windows, activation de la virtualisation, réinstallation de Docker Desktop, etc.).
\end{warningbox}

\subsection{Étape 2 -- Vérification de Git}

\begin{enumerate}[leftmargin=*, label=\textbf{2.\arabic*}]
    \item Dans le même terminal, vérifiez la présence de Git :
\begin{lstlisting}[language=bash,caption={Vérification de Git}]
git --version
\end{lstlisting}
    \item Si la commande échoue, installez Git depuis : \url{https://git-scm.com/downloads}, puis recommencez la vérification.
\end{enumerate}

\subsection{Étape 3 -- Clonage des deux repositories}

Choisissez ou créez un dossier de travail (par exemple : \texttt{C:\textbackslash middleware-lab} ou \texttt{~/middleware-lab}). Puis exécutez les commandes suivantes :

{\footnotesize
\begin{lstlisting}[language=bash,breaklines=false,basicstyle=\ttfamily\footnotesize,caption={Clonage des projets du TP}]
# Se placer dans votre dossier de travail
cd chemin/vers/votre/dossier

# 1) Architecture microservices SANS API Gateway
git clone https://github.com/mouadd9/demo1-microservices-without-a-gateway.git

# 2) Architecture microservices AVEC API Gateway
git clone https://github.com/mouadd9/demo2-microservices-with-a-gateway.git
\end{lstlisting}
}

Pour consultation directe du code source et de la documentation en ligne, vous pouvez aussi accéder aux dépôts GitHub :
\begin{itemize}[leftmargin=*]
    \item \url{https://github.com/mouadd9/demo1-microservices-without-a-gateway}
    \item \url{https://github.com/mouadd9/demo2-microservices-with-a-gateway}
\end{itemize}

Vous devez obtenir deux dossiers :
\begin{itemize}[leftmargin=*]
    \item \texttt{demo1-microservices-without-a-gateway}
    \item \texttt{demo2-microservices-with-a-gateway}
\end{itemize}

Pour être cohérent avec les chemins utilisés dans le \textbf{guide TP principal}, vous pouvez \textbf{renommer manuellement} ces dossiers (clic droit \textrightarrow{} Renommer) en :
\begin{itemize}[leftmargin=*]
    \item \texttt{microservices-without-a-gateway}
    \item \texttt{microservices-with-a-gateway}
\end{itemize}

\begin{successbox}[À ce stade]
Les sources des deux architectures sont téléchargées en local sur votre machine et prêtes à être utilisées par Docker Compose.
\end{successbox}

\newpage

\subsection{Étape 4 -- Construction des images Docker (sans Gateway)}

Dans cette étape, nous construisons toutes les images nécessaires pour \textbf{l'architecture sans middleware d'entrée (sans API Gateway)}. Cette première version sert de point de comparaison pour illustrer les \textbf{limites d'un système distribué sans middleware}, avant d'introduire une topologie régulée.

\begin{enumerate}[leftmargin=*, label=\textbf{4.\arabic*}]
    \item Placez-vous dans le dossier du premier projet (en adaptant le chemin en fonction de votre organisation) :
\begin{lstlisting}[language=bash]
cd microservices-without-a-gateway
\end{lstlisting}

    \item Construisez les images :
\begin{lstlisting}[language=bash,caption={Construction des images pour l'architecture sans Gateway}]
# Utilisez l'une des deux syntaxes selon votre version
Docker compose build
# ou
Docker-compose build
\end{lstlisting}

    \item Patientez jusqu'à la fin de la construction (les premières fois, Docker télécharge les couches de base et installe les dépendances).

    \item Vérifiez la présence des images dans votre cache Docker :
\begin{lstlisting}[language=bash]
docker images
\end{lstlisting}
\end{enumerate}

\begin{infobox}[Astuce]
Si vous avez le temps, vous pouvez lancer une fois l'architecture complète pour vérifier que tout démarre correctement :
\begin{lstlisting}[language=bash]
Docker compose up
# ou: docker-compose up
\end{lstlisting}
Puis arrêtez et nettoyez :
\begin{lstlisting}[language=bash]
Docker compose down
# ou: docker-compose down
\end{lstlisting}
Le jour du TP, le redémarrage sera alors quasi immédiat, car toutes les images sont déjà construites.
\end{infobox}

\subsection{Étape 5 -- Construction des images Docker (avec Gateway)}

Répétez la même logique pour \textbf{l'architecture avec middleware d'entrée (API Gateway)}.\\
Dans le cadre de cet atelier, la Gateway est utilisée comme \textbf{exemple concret de middleware} permettant de réguler une architecture microservices. Un \textbf{autre groupe} proposera un atelier entièrement dédié à l'API Gateway elle-même (patterns avancés, configuration détaillée, etc.).

\begin{enumerate}[leftmargin=*, label=\textbf{5.\arabic*}]
    \item Revenez au dossier parent puis placez-vous dans le deuxième projet :
\begin{lstlisting}[language=bash]
cd ..
cd microservices-with-a-gateway
\end{lstlisting}

    \item Construisez les images correspondantes :
\begin{lstlisting}[language=bash,caption={Construction des images pour l'architecture avec Gateway}]
Docker compose build
# ou: docker-compose build
\end{lstlisting}

    \item Optionnel : testez un démarrage complet, puis arrêtez :
\begin{lstlisting}[language=bash]
Docker compose up
Docker compose down
\end{lstlisting}
\end{enumerate}

\begin{successbox}[Bilan de la préparation]
Si toutes les étapes précédentes sont complétées :
\begin{itemize}[leftmargin=*]
    \item Docker Desktop fonctionne et peut exécuter des conteneurs ;
    \item Git est installé et les deux repositories sont clonés ;
    \item les images Docker des deux architectures (sans middleware d'entrée et avec middleware d'entrée) sont déjà construites ;
    \item le jour de la séance, vous n'aurez plus qu'à exécuter les commandes de lancement indiquées dans le \textbf{guide TP principal} (\og docker-compose up \fg{}), sans attendre le téléchargement ni la construction des images, et vous pourrez vous concentrer sur la \textbf{comparaison des topologies de middleware} (API Gateway n'étant qu'un exemple parmi d'autres).
\end{itemize}
Vous êtes donc prêt(e) à suivre la \textbf{simulation en temps réel} dans des conditions optimales.
\end{successbox}

\end{document}


